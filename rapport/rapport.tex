\documentclass[a4paper, titlepage, oneside, 12pt]{article}%      autres choix : book  report

\usepackage[utf8]{inputenc}%           gestion des accents (source)

\usepackage[T1]{fontenc}%              gestion des accents (PDF)

\usepackage[francais]{babel}%          gestion du français

\usepackage{textcomp}%                 caractères additionnels

\usepackage{mathtools,  amssymb, amsthm}% packages de l'AMS + mathtools

\usepackage{lmodern}%                  police de caractère

\usepackage{geometry}%                 gestion des marges

\usepackage{graphicx}%                 gestion des images

\usepackage{xcolor}%                   gestion des couleurs

\usepackage{array}%                    gestion améliorée des tableaux

\usepackage{calc}%                     syntaxe naturelle pour les calculs

\usepackage{titlesec}%                 pour les sections

\usepackage{titletoc}%                 pour la table des matières

\usepackage{fancyhdr}%                 pour les en-têtes

\usepackage{titling}%                  pour le titre

\usepackage[framemethod=TikZ]{mdframed}% print frames

\usepackage{caption}%                  for captionof

\usepackage{listings}

\usepackage{enumitem}%                 pour les listes numérotées

\usepackage{microtype}%                améliorations typographiques

\usepackage{csvsimple}%                 convertir un fichier .csv en tableau

\usepackage{hyperref}%                 gestion des hyperliens

\hypersetup{pdfstartview=XYZ}%         zoom par défaut

%%%%%%%%%%%%%%%%%%%%%% CONFIGURATION %%%%%%%%%%%%%%%%%%%%%%%%%%

\mdfdefinestyle{MyFrame}{%
    linecolor=black,
    outerlinewidth=0pt,
    roundcorner=10pt,
    innertopmargin=\baselineskip,
    innerbottommargin=\baselineskip,
    innerrightmargin=20pt,
    innerleftmargin=20pt,
    backgroundcolor=black!10!white}

\captionsetup[lstlisting]{labelformat=empty}

%%%%%%%%%%%%%%%%%%%%%%%%%%%%%%%%%%%%%%%%%%%%%%%%%%%%%%%%%%%%%%%

%%%%%%%%%%%%%%%%%%%%%% COMMANDES PL %%%%%%%%%%%%%%%%%%%%%%%%%%
\newcommand\boldmin{\mathop{\mathbf{min}}}
\newcommand\boldmax{\mathop{\mathbf{max}}}



% I don't know how it works but it does !
\def\foo#1#2{%
\newenvironment{variables}
{\paragraph{Variables :}
#1{2}}
{#2}}
\expandafter\foo
  \csname alignat*\expandafter\endcsname
  \csname endalignat*\endcsname


% \newenvironment{variables}
% {  \paragraph{Variables :}
  
%   \alignat{2} }
% {  \endalignat }

\newenvironment{fonctionobj}
{ \paragraph{Fonction objectif :}


  }
{  }

\def\cons#1#2{%
\newenvironment{contraintes}
{\paragraph{Contraintes :}
#1{2}}
{#2}}
\expandafter\cons
  \csname alignat*\expandafter\endcsname
  \csname endalignat*\endcsname


\newcommand{\variable}[4]{\underbrace{#1}_{\mathclap{\text{#4}}} \in #2 &\ #3 \\}
\newcommand{\fobj}[2]{\begin{alignat*}{2} #1 &\qquad \text{\footnotesize (#2)} \end{alignat*}}
\newcommand{\constraint}[3]{#1 &,\ & #2 & 
\if\relax\detokenize{#3}\relax
\\
\else
\qquad \text{\footnotesize \textcolor{blue}{\textit{#3}}} \\ 
\fi}
%%%%%%%%%%%%%%%%%%%%%%%%%%%%%%%%%%%%%%%%%%%%%%%%%%%%%%%%%%%%%%%




\title{Rapport pour le projet de MOGPL}
\date{\today}
\author{Pierre Mahé \& François Thiré}

 % Modèle pour écrire un programme linéaire
 % \begin{alignat*}{2}
 %    \text{minimize }   & \sum_{i=1}^m c_i x_i + \sum_{j=1}^n d_j t_j\  \\
 %    \text{subject to } & \sum_{i=1}^m a_{ij} + e_j t_j \geq g_j &,\ & 1\leq j\leq n\\
 %                       & f_i x_i + \sum_{j=1}^n b_{ij}t_j \geq h_i\ &,\ & 1\leq i\leq m\\
 %                       & x\geq 0,\ t_j\geq 0\ &,\ & 1\leq j\leq n,\ 1\leq i\leq m
 %  \end{alignat*}


\begin{document}

\maketitle

\section{Première modélisation du problème}
\subsection{Question 1}



\begin{mdframed}[style=MyFrame]


\begin{variables}
\variable{x_{i,j}}{\mathbb{B}}{1 \leq i,j \leq n}{Vaut $1$ si l'agent $a_i$ reçoit le bien $b_j$}
\end{variables}

\begin{fonctionobj}

\fobj{\boldmax \ \frac{1}{n}\ \sum_{\substack{1\leq i,j \leq n}} u_{i,j}x_{i,j}}
{avec $u_{i,j}$ les coefficients d'utilités du bien $i$ pour l'agent $j$}
\end{fonctionobj}

\begin{contraintes}
   \constraint{\sum_{i=1}^n x_{i,j} =1}{ 1\leq j\leq n}{$1$ agent par bien}
   \constraint{\sum_{j=1}^n x_{i,j} =1}{ 1\leq i\leq n}{$1$ bien par agent}
\end{contraintes}

\end{mdframed}
\captionof{lstlisting}{Programme linéaire $\mathcal{P}_0$}

\subsection{Question 2}
Vous trouverez le code \textit{Python} du modèle \textit{PO}  dans le répertoire \textit{Python/modelisation\_P0}. 
L'implémentation du modèle et la génération des tests se fait dans le fichier \textbf{P0.py}. Pour connaître les options possibles du programme, veuillez lancer la commande

%TO DO : faire un meilleur rendu
\begin{lstlisting}[language=bash]
gurobi.sh P0.py -h
\end{lstlisting}

En particulier pour lancer le programme avec comme taille $n=100$ et $M=1000$ on utilisera la commande suivante :
\begin{lstlisting}[language=bash]
gurobi.sh P0.py -n 100 -M 1000
\end{lstlisting}

Si vous choisissez d'utiliser les options pour enregistrer le modèle et écrire la solution dans un fichier, alors le programme va créer respectivement deux sous-dossiers\footnote{à partir du dossier courant} \textit{modele} et \textit{solutions} qui contiendra ces fichiers.

\subsection{Question 3}

Nous avons automatisé la création des tableaux par un script \textit{bash} que vous trouverez dans le répertoire \textit{Python/modelisation\_P0/stats}. Ce programme va générer des fichiers \textit{csv} qui pourront ensuite être importés dans un fichier \LaTeX en utilisant le package \textit{csvsimple}. Voici donc les résultats trouvés :

\begin{table}
\csvautotabular{csv/data_10.csv}
\caption{Résultats lorsque $M=10$}
\end{table}

\begin{table}
\csvautotabular{csv/data_100.csv}
\caption{Résultats lorsque $M=100$}
\end{table}

\begin{table}
\csvautotabular{csv/data_1000.csv}
\caption{Résultats lorsque $M=1000$}
\end{table}
\section{Approche égalitariste}
\subsection{Question 4}
\paragraph{Description du graphe : } On décrit dans ce paragraphe la construction du graphe lié au problème. 

Soit $n$ le nombre d'agents et le nombre de bien à répartir. Et soit $u_{i,j}$ l'utilité du bien $j$ pour l'agent $i$. 
On fait l'hypothèse que les coefficient de la mtrice $u_{i,j}$ soient distincts\footnote{L'hypothèse n'est pas très forte car il suffit sinon de tirer au hasard un $\varepsilon>0$ pour différencier les coefficients égaux}. 
On construit le graphe orienté suivant :
$G=(V,E)$ où 
\begin{itemize}
\item $V=\{s,u_1,_u2,\dots, u_n, b_1,b_2,\dots, b_n,p\}$
\item $\forall i,j \in \{1,\dots, n\} $
  \begin{itemize}
  \item $(u_i,b_j)\in E $
  \item $(s,u_i)\in E$
  \item $(b_j,p) \in E$
  \end{itemize}
\end{itemize}  

Il reste à décrire les capacités sur chaque arc.  $\forall i,j \in \{1, \dots, n\}$
\begin{itemize}
\item $c(u_i,b_j)=u_{i,j}-\lambda$
\item $c(s,u_i)=\min \{c(u_i,b_j) \mid c(u_i,b_j)\geq 0\}$
\item $c(b_j,b_j)=\min\{c(u_i,b_j) \mid c(u_i,b_j)\geq 0\}$
\end{itemize}

\paragraph{Justification de la construction du graphe : }


\section{Approche égalitariste en regrets}
\section{Extension à l'affectation multiple}
\end{document}